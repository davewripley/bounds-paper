\documentclass{article}
\usepackage{amssymb, amsthm}

\newcommand{\G}{\ensuremath{\Gamma}}
\newcommand{\Sig}{\ensuremath{\Sigma}}
\newcommand{\pcup}{\ensuremath{\sqcup}}
\newcommand{\pos}[2]{\ensuremath{[#1:#2]}}

\newcommand{\cns}{\vdash}
\newcommand{\ded}{\Vdash}

\newtheorem{lem}{Lemma}

\title{From conversation to inference, via consequence}
\author{Greg Restall and David Ripley\\Universities of Melbourne and Connecticut\\restall@unimelb.edu.au and davewripley@gmail.com}
\begin{document}

\maketitle

\section{Intro}

In a number of papers (\cite{restall:tvpt, restall:cf2d, restall:adnct, ripley:pafc, ripley:bcw}, among others), we have applied an approach to consequence first explored in \cite{restall:mc}.
Our goal here is to explore and make plain the underpinnings of this approach to consequence, and to use the approach to draw out the relations between conversational practice and correct deductive inference.

\section{Conversation}

This approach is based on the idea of {\em bounds} on {\em positions}.
Both of these find their home in our conversational practices.

\subsection{Positions}

Scoreboard model.
Need to track each other's assertions and denials.
Need to do the same under suppositions. 

\subsection{Bounds}

Some positions are {\em in bounds} and some not.
Stance approach: this is a case where treating a position as out of bounds makes it so.
(As usual, this doesn't prevent all mistakes, only a certain kind of systematic and widespread mistake.)

\subsubsection{Social kinds}

Being out of bounds is like being queen.
It's a status that's had by someone in virtue of the role they play in certain social practices.

\subsubsection{Treating as out of bounds} \label{treating}

What, then, are the practices, and the roles in them, that make positions in bounds or not?
Dismissal, reinterpretation, asking for clarification, reductio arguments. Others? 

[Question from GR: what's the connection between this evaluation and our identification of same-saying? I've used the rhetoric that treating my saying ``$X$'' to say the same thing as your utterance ``$Y$'' is tied up with treating my assertion of ``$X$'' and your denial of ``$Y$'' as being out of bounds---there's no position containing both my assertion and your denial. I think this is a part of what is involved in the treating of things as in bounds and out of bounds, but I'm not sure where in the explanation it might go.]

[[Response from DR: What about flipping it? Asserting $X$ can clash with denying $Y$ even when they don't say the same, but treating an assertion of $X$ as {\em compatible} with a denial of $Y$ is pretty conclusively treating them as meaning something different. Could this go in \S\ref{identity} (newly added)? I'm happy to leave handling this issue to you; I think you've thought more about it.]]

\section{Consequence} \label{consequence}

\subsection{Consequence from bounds}

We adopt the strategy of \cite{restall:mc} for developing a consequence relation (multiple-premise, multiple-conclusion) from the bounds.

This section, then, largely reviews things covered there (and elsewhere).


\subsection{Structural properties}

What can be said without appeal to any particular vocabulary?


\subsubsection{Differences in meaning} \label{identity}

One thing we can say immediately is that assertion and denial of the very same thing clash with each other. 
This immediately gives a sufficient condition for difference in meaning: if the position that asserts $A$ and denies $B$ is in bounds, then $A$ and $B$ don't mean the same.
Since anything whatsoever means the same as itself, this gives us $A \cns A$, for every $A$; it is out of bounds to assert and deny the very same thing, whatever that thing is.


\subsubsection{Subpositions} \label{weakening}

Look at ways in which the practices identified in \S\ref{treating} work to ensure that adding to something out of bounds leaves it out of bounds.

\subsubsection{Extensibility} \label{cut}

Extensibility discussion as in our other pieces.
New way to frame it: a position's {\em extensions} are the in-bounds positions that contain it. 
Given closure principles on bounds, we can describe the maximum closed reduction to a position's extensions that doesn't allow for $A$ to be denied.
Question: does adding an assertion of $A$ always bring us to this maximum closed reduction?
(Denial and assertion reversed, too, of course.)

Return to practices of \S\ref{treating}; what aspects of them are at stake here?

\subsection{Meaning} \label{meaning}

Any way of understanding logical constants (or anything else) as getting their meaning from rules on a consequence relation needs some interpretation of the consequence relation in question.
Using the present interpretation for this purpose, meaning comes out as a matter of assertion and denial conditions.
This is related to inferentialist views of meaning, but not exactly inferentialist itself, since inference isn't on stage yet.

\subsubsection{Examples: $\wedge$ and $\neg$} \label{operational}




\section{Inference}

What, then, is the connection between an argument's being valid on the one hand, and the act of {\em inferring} on the other?
\cite{steinberger:wcsrs}, for example, insists on what he calls the {\em principle of answerability:} `[O]nly such deductive systems are permissible as can be seen to be suitably connected to our ordinary deductive inferential practices'.
Of course Steinberger does not mean that it is {\em immoral} to conduct research into other deductive systems; the permissibility in question here is permissibility for use in inferentialist theories of meaning, which see the meanings of at least the logical constants as deriving from their roles in correct inference.

He uses this principle to argue against the kind of consequence relation considered in \S\ref{consequence} serving as the basis for such an inferentialist theory.
Indeed, as we've mentioned, the kind of meaning theory we offer in \S\ref{meaning} {\em isn't} exactly inferentialist, although it bears considerable resemblance to inferentialist theories.
Still, there is room to wonder about the connection between this consequence relation and `our ordinary deductive inferential practices'.

\subsection{Deduction as non-ampliative}

Deductive inference is meant to be, in some sense, {\em non-ampliative}.
We should not be able to validly deduce our way to things that are not already present in our starting point.
Exactly what this means, though, is up for grabs.
We want to offer a reading based on the notion of {\em implicit assertion}.
The key idea will be this: a inference from $\G$ to $A$ is non-ampliative, and so deductively sanctioned, iff any position that asserts everything in $\G$ {\em implicitly asserts} $A$.
This is one important sense in which $A$ does not go beyond what is already in $\G$.
We write `$\G \ded A$' to mean that the inference from $\G$ to $A$ is non-ampliative in this sense.\footnote{This kind of deductive inference is biased towards assertion rather than denial; it has a twin, biased towards denial instead. 
Just as an assertion may be implicit in a body of other assertions, so may a denial be implicit in a body of other denials. 
These twins, in turn, are part of a larger family: assertions might be implicit in collections of denials, denials in collections of assertions, and both assertions and denials in collections that combine assertions and denials.
Here, we explore only the `assertion-assertion' case, leaving the others for other work.}

\subsubsection{Implicit assertion}

Making this idea precise requires a precise notion of implicit assertion itself.
Here's one.

Every position $P$ has options.
A position $Q$ is an {\em option} for a position $P$ iff $P \pcup Q$ is in bounds.
(The option relation is thus symmetric, and any out-of-bounds position has no options.)
A position $P$ implicitly asserts $A$ iff it has the same options as the position $P \pcup \pos{A}{}$.
That is, adding an assertion of $A$ to $P$ wouldn't change the options for $P$.
This is, we think, a natural notion of implicit assertion, and it is the one we will draw on in what follows.\footnote{\label{fn-compact-1}Our notion of non-ampliative inference moves from {\em actually} asserting a set $\G$ of premises to {\em implicitly} asserting a conclusion $A$.
There is a related notion, which would move from {\em implicitly} asserting each member of $\G$ to implicitly asserting its conclusion.
These are in general different: there is no guarantee, when an in-bounds position $P$ implicitly asserts each member of $\G$, that $P \pcup \pos{\G}{}$ is in bounds at all, let alone that it has the same options as $P$.
However, this difference depends on interactions between infinite $\G$ and noncompact bounds; for finite $\G$ or compact bounds the ideas collapse into each other.
We do not consider this difference further here, except for a comment in footnote \ref{fn-compact-2}.
}


\subsubsection{The shape of deduction}

Here, we explore some aspects of this non-ampliative relation.
First, it is reflexive: any position that actually asserts $A$ also implicitly asserts it, on the given understanding, and so $A \ded A$. Second, it is monotonic: if $\G \subseteq \G'$, then any position that asserts the $\G'$s also asserts the $\G$s. So if all the latter positions implicitly assert $A$, so do all the former: if $\G \ded A$, then $\G' \ded A$.

Third, it is transitive, in the following sense: if $\G \ded A$, and $\G', A \ded B$, then $\G, \G' \ded B$.\footnote{\label{fn-compact-2}There is a stronger transitivity property sometimes considered that is not guaranteed to hold here: we might have $\G \ded A$ for each $A \in \Sig$ and $\G', \Sig \ded B$ without $\G, \G' \ded \Sig$.
The issue, as in footnote \ref{fn-compact-1}, has to do with infinite sets (this time $\Sig$) and noncompact bounds; we do not consider it further.
}
To see this, suppose the antecedent and consider an arbitrary position $P$ that asserts $\G \cup \G'$.
Since $\G \ded A$, it must be that $P$ implicitly asserts $A$; it thus has the same options as $Q = P \pcup \pos{A}{}$. 
But $Q$ asserts $\G'$ and $A$, so since $\G', A \ded B$, it must be that $Q$ implicitly asserts $B$; it thus has the same options as $Q \pcup \pos{B}{}$.
This means that $P$ has the same options as $Q \pcup \pos{B}{} = P \pcup \pos{A, B}{}$.
But if $P$ has the same options as $P \pcup \pos{A, B}{}$, it must have the same options as $P \pcup \pos{B}{}$, by the subposition property argued for in \S\ref{weakening}.
So $P$ implicitly asserts $B$; since $P$ was arbitrary, $\G, \G' \ded B$.

There is an interesting phenomenon visible here: the structural properties of non-ampliative inference do not depend in any straightforward way on the structural properties of bounds consequence.
We have shown here that $\ded$ is reflexive and monotonic {\em without any appeal at all} to the nature of the bounds themselves.
Any bounds whatever, even bounds that violate the conditions we've argued for in \S\ref{identity} and \S\ref{weakening}, will yield a $\ded$ that is reflexive and monotonic.
Similarly, we've shown that $\ded$ is transitive without any appeal to the extensibility condition of \S\ref{cut}; it is instead the {\em subposition} condition of \S\ref{weakening} that matters.\footnote{The subposition condition is indeed needed. Proof of this (if it's true!) goes here.}
Real caution is needed, then, about seeing a framework like the present one as `monotonic' or not, `transitive' or not, and so on. 
It is certain relations that do or do not have these properties; $\cns$ and $\ded$ are sensitive to different aspects of the bounds, and so need not exhibit these properties in lockstep with each other.

Similarly, we can explore how the connectives $\wedge$ and $\neg$ considered in \S\ref{operational} relate to $\ded$.


\subsection{Inference and extensibility}

Above, we've considered only what we could see about deduction while remaining neutral on the issue of extensibility, considered in \S\ref{cut}.
<<<<<<< Updated upstream
But extensibility turns out to matter a great deal for the relation between $\cns$ and $\ded$.
In fact, given extensibility, they turn out to be the same, at least for arguments with a single conclusion. 
Here, we show this.

\begin{lem}
If $\G \ded A$, then $\G \cns A$.
\end{lem}

\begin{proof}
Suppose $\G \ded A$. 
We need to show the position $\pos{\G}{A}$ is out of bounds.
This position asserts $\G$, so it implicitly asserts $A$. 
That is, it has the same options as $\pos{\G, A}{A}$.
But $\pos{\G, A}{A}$ {\em is} out of bounds; it both asserts and denies $A$.
So the empty position is not an option for it.
Neither, then, it is an option for $\pos{\G}{A}$; but that is just to say that $\pos{\G}{A}$ is out of bounds.
\end{proof}

\begin{lem}
If $\G \cns A$ and the bounds obey extensibility, then $\G \ded A$.
\end{lem}

\begin{proof}
Suppose $\G \cns A$, and suppose the bounds obey extensibility.
Take any position $P$ that asserts $\G$; we need to show that $P$ has the same options as $P \pcup \pos{A}{}$.
Clearly it has no {\em fewer} options, by the subposition condition; so we need only show that it has no {\em more}.
Suppose, then, that $\pos{\Sig}{\Theta}$ is an option for $P$.
Then $P \pcup \pos{\Sig}{\Theta}$ is in bounds.
By extensibility, either $P \pcup \pos{\Sig, A}{\Theta}$ or $P \pcup \pos{\Sig}{\Theta, A}$ is in bounds.
But $P \pcup \pos{\Sig}{\Theta, A}$ can't be in bounds: this position asserts $\G$ (since $P$ does) and denies $A$, and we know $\pos{\G}{A}$ is out of bounds.
By the subposition condition, then, $P \pcup \pos{\Sig}{\Theta, A}$ is also out of bounds.
So $P \pcup \pos{\Sig, A}{\Theta}$ is in bounds.
That is, $\pos{\Sig}{\Theta}$ is an option for $P \pcup \pos{A}{}$.
\end{proof}

Note that this is perfectly general as to what kind of vocabulary is present or absent.
Whatever shape $\cns$ has, if the bounds obey identity, subposition, and extensibility, this is enough to show that $\ded$ is the single-conclusion fragment of $\cns$.

=======

Here, we show that {\em given extensibility}, the full consequence relation of \S\ref{consequence} turns out to be applicable to deductive inference.\footnote{That is, we show it if it's true! It smells true, though.}
>>>>>>> Stashed changes


\section{Conclusion}




\bibliographystyle{apalike}
\bibliography{cci}

\end{document}

%% 
